\section{Hardware dell'adapter board}
L'hardware utilizzato per condurre lo studio oggetto di questa tesi consiste in due piattaforme indossabili per effettuare misure fotopletismografiche. Ciascuna piattaforma si compone di una \textit{Adapter Board} e una scheda ospitante un microcontrollore. L'Adapter Board consiste in una PCB sulla quale sono montati il modulo PPG, un accelerometro ed, eventualmente, un regolatore di tensione lineare(LDO) per l'alimentazione dei componenti. Il microcontrollore viene utilizzato per l'acquisizione dei dati dal sensore e per il suo controllo. In particolare, è stata utilizzata in entrambe le soluzioni la board \textbf{STM32F4DISCOVERY}, prodotta da STMicroelectronics, ospitante il microntrollore \textbf{STM32F407}. Grazie poi ad un'interfaccia USB è possibile trasmettere i dati ad un computer per permetterne l'elaborazione e la memorizzazione.
Le due piattaforme si differenziano per l'Adapter Board, in cui sono stati utilizzati due differenti moduli PPG: il \textbf{MAXM86161} e \textbf{MAX86916}, prodotti entrambi da Maxim Integrated.
\subsection{Adapter Board: MAXM86161}
\begin{figure}[b]
	\centering
	\includegraphics[width=0.6\linewidth]{ImageFiles/Hardware/DiagrammaBlocchiMAXM86161}
	\caption{Diagramma a bloccchi dell'Adapter Board con modulo PPG MAXM86161}
	\label{fig:DiagrammaBlocchiMAXM86161}
\end{figure}
L'Adapter Board progettata ospita il sensore PPG MAXM96161 e l'accelerometro triassiale LIS2DW12, che comunicano con il microcontrollore tramite protocollo I\ap{2}C (\Fig~\ref{fig:DiagrammaBlocchiMAXM86161}). Grazie al numero ridotto di componenti è stato possibile ottenere una scheda dalle dimensioni molto ridotte (12,4 mm x 4,6 mm).

\paragraph{Sensore PPG} Il sensore PPG utilizzato è il MAXM86161 rappresentato nella figura \ref{fig:ImmagineMAXM86161}.
\begin{figure}[tb]
	\centering
	\includegraphics[width=0.7\linewidth]{ImageFiles/Hardware/ImmagineMAXM86161}
	\caption{Il modulo MAXM86161.}
	\label{fig:ImmagineMAXM86161}
\end{figure}
Si tratta di un modulo integrato a basso consumo, per acquisizioni di dati ottici. Il sensore integra 3 LED (rosso, infrarosse e verde), un fotodiodo ed elementi ottici. All'interno è anche presente un regolatore di tensione lineare (LDO). Per questo motivo non è necessario inserire un ulteriore LDO esterno per l'alimentazione dei circuiti interni e dei LED. Il modulo viene alimentato a singola tensione, che deve essere compresa tra i 3.0V e 5.5V. \`E presente un pin di uscita (VLDO) collegato all'uscita del regolatore interno che fornisce una tensione di 1.8V che può essere utilizzato per alimentare eventuali dispositivi esterni. Infatti, nell'Adapter Board, questa uscita è stata utilizzata per alimentare l'accelerometro. 
La comunicazione con il microcontrollore avviene tramite i pin SDA e SCL, tipici della comunicazione I\ap{2}C. Il modulo presenta un package di tipo OLGA a 14 pin, dalle dimensioni ridotte, 2.9 mm x 4.3 mm x 1.4 mm, e un consumo minore di \SI{30}{\micro\watt}.

\paragraph{Accelerometro} Il LIS2DW12 è un accelerometro triassiale lineare, prodotto da STMicroelectronics\cite{STElectronicsLIS2DW12}, realizzato con tecnologia MEMS ad alte performance e basso consumo. 
\begin{figure}[b]
	\centering
	\includegraphics[width=0.3\linewidth]{ImageFiles/Hardware/ImmagineLIS2DW12}
	\caption{Accelerometro LIS2DW12.}
	\label{fig:ImmagineLIS2DW12}
\end{figure}
\`E pensato per rilevazioni di movimento e riconoscimento dei gesti su dispositivi indossabili. Il sensore permette di selezionare un fondo scala di $\pm$\SI{2}{\gram}, $\pm$\SI{4}{\gram}, $\pm$\SI{8}{\gram} oppure $\pm$\SI{16}{\gram}. Necessita di un'alimentazione compresa tra i 1.62V e i 3.6V e la corrente assorbita tipicamente è di \SI{50}{\nano\ampere} nella modalità \textit{power-down} e di \SI{1}{\micro\ampere} in quella attiva \textit{low-power}. Inoltre, l'accelerometro include un sensore di temperatura che permette di compensare gli effetti termici sulle misure. Presenta un'interfaccia I\ap{2}C e un interfaccia SPI per la comunicazione. Questo sensore è stato inserito nell'Adapter Board per ottenre una stima di eventuali disturbi, dovuti agli artefatti del movimento, permettendo quindi di ottenere delle stime dei parametri di qualità superiore. Il sensore è realizzato in un package di tipo LGA-12 con dimensioni 2.0 mm x 2.0 mm x 0.7 mm.
\todo{tabella riassuntiva dei componenti come facagni}

\paragraph{Progetto PCB} Il progetto della board è stato realizzato con il software Autodesk Eagle. Come prima fase del progetto è stato realizzato lo schematico, dove sono state definite le interconnessioni dei componente a livello logico. In seguito, è stato definito il layout della PCB, definendo la posizione dei componenti e delle piste che li connettono.
In figura \ref{fig:ImmagineMAXM86161} viene riportato lo schematico dove si possono vedere il sensore PPG, l'accelerometro e due connettori necessari, uno per l'alimentazione della scheda (quello sopra)\todo{non molto bello}, e uno per la comunicazione I\ap{2}C (quello sotto\todo{non molto bello magari c'è il modo per definirlo in modo più elegante}).
\begin{figure}[b]
	\centering
	\includegraphics[width=0.8\linewidth]{ImageFiles/Hardware/schematic_maxm}
	\caption{Schematico Adapater Board con il sensore PPG MAXM86161}
	\label{fig:schematic_maxm}
\end{figure}
\todo{attenzione a posizionare sempre le immagini o b o t come vuole traversi.. in questo caso mi sembra meglio posizionare entrambe a b ma anche t e b non è male}
Il layout è stato progettato partendo dallo schematico, avendo cura di realizzare una board dalle dimensioni più ridotte possibili. Come si può vedere in figura \ref{fig:Layout_maxm} la board si compone su due layer, Top (in rosso) e Bottom (in blu).
\todo{Immagini da sistemare per metterle in fila a stessa dimensione}
\begin{figure}[b]
	\centering
	\includegraphics[width=0.15\linewidth]{ImageFiles/Hardware/layout_top_maxm}
	\includegraphics[width=0.15\linewidth]{ImageFiles/Hardware/layout_bottom_maxm}
	\includegraphics[width=0.15\linewidth]{ImageFiles/Hardware/manifacturing_top_maxm}
	\includegraphics[width=0.15\linewidth]{ImageFiles/Hardware/manifacturing_bottom_maxm}
	\caption{Da sistemare}
	\label{fig:Layout_maxm}
\end{figure}
Le piste disegnate hanno una larghezza di 0.2 mm e dei fori con diametro di 0.3 mm, con una corona di \SI{254}{\micro\meter}.

\pagebreak
\subsection{Adapter Board: MAX86916}

\paragraph{Sensore PPG} Il sensore PPG utilizzato è il MAX86916, prodotto da Maxim Integrated, descritto precedentemente come stato dell'arte.

\begin{figure}[tb]
	\centering
	\includegraphics[width=0.6\linewidth]{ImageFiles/Hardware/ImmagineMAX86916}
	\caption{MAX86916 magari mettere in evidenza led e foto diodo}
	\label{fig:ImmagineMAX86916}
\end{figure}

\paragraph{LDO} \todo{Parliamo dei 3 LDO che abbiamo valutato? si parlerei di tutti e tre o almeno 2 mettendo in evidenza magari una tabella con le differenze}

\paragraph{Accelerometro}\todo{ut supra} L'accelerometro utilizzato è sempre \textbf{LIS2DH12} prodotto da STMicroelectronics.

\subsection{Microcontrollore: STM32F4DISCOVERY}\todo{farei una descrizione con alcune nozione sui componenti}
\todo{inserire immagine della board}