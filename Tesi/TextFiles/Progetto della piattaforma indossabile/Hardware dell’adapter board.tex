\section{Hardware dell'adapter board}
L'hardware utilizzato per condurre lo studio, oggetto di questa tesi, consiste in due piattaforme indossabili per effettuare misure fotopletismografiche. Ciascuna piattaforma si compone di una \textit{Adapter Board} e una scheda sulla quale è montato un microcontrollore. L'Adapter Board è costituita da una PCB sulla quale sono montati il modulo PPG, un accelerometro e, in una delle due board, un regolatore \textit{low-dropout} di tensione (LDO) per l'alimentazione dei componenti. Il microcontrollore viene utilizzato per l'acquisizione dei dati dai sensori e per il loro controllo. In particolare, è stata utilizzata in entrambe le soluzioni la board \textbf{STM32F4DISCOVERY}, prodotta da STMicroelectronics, che ospita un microntrollore \textbf{STM32F407}. Inoltre, è possibile trasmettere i dati acquisiti ad un computer grazie ad un'interfaccia USB. In questo modo sarà possibile salvare ed elaborare i dati raccolti.
Le due piattaforme si differenziano nell'Adapter Board progettata, in cui sono stati utilizzati due differenti moduli PPG: il \textbf{MAXM86161} e \textbf{MAX86916}, prodotti entrambi da Maxim Integrated.
\subsection{Adapter Board: MAXM86161}
\begin{figure}[b]
	\centering
	\includegraphics[width=0.6\linewidth]{ImageFiles/Hardware/DiagrammaBlocchiMAXM86161}
	\caption{Diagramma a bloccchi dell'Adapter Board con modulo PPG MAXM86161}
	\label{fig:DiagrammaBlocchiMAXM86161}
\end{figure}
L'Adapter Board progettata ospita il sensore PPG MAXM96161 e l'accelerometro triassiale LIS2DW12, che comunicano con il microcontrollore tramite protocollo I\ap{2}C (\Fig~\ref{fig:DiagrammaBlocchiMAXM86161}). Grazie al numero ridotto di componenti è stato possibile ottenere una scheda dalle dimensioni molto ridotte (12,4 mm x 4,6 mm), con una superficie di appena \SI{57.04}{\square\milli\meter}.

\paragraph{Sensore PPG} Il sensore PPG utilizzato è il MAXM86161 rappresentato nella figura \ref{fig:ImmagineMAXM86161}.
\begin{figure}[tb]
	\centering
	\includegraphics[width=0.7\linewidth]{ImageFiles/Hardware/ImmagineMAXM86161}
	\caption{Il sensore MAXM86161.}
	\label{fig:ImmagineMAXM86161}
\end{figure}
Si tratta di un modulo integrato a basso consumo per acquisizioni di dati ottici. Il sensore integra 3 LED (rosso, infrarosse e verde), un fotodiodo ed elementi ottici. All'interno è anche presente un regolatore di tensione di tipo LDO. Per questo motivo non è necessario inserire un ulteriore LDO esterno per l'alimentazione dei circuiti interni e dei LED. Il modulo viene alimentato con una singola tensione che deve essere compresa tra i 3.0V e 5.5V. \`E presente un pin di uscita (VLDO) collegato all'uscita del regolatore interno che fornisce una tensione stabile di 1.8V. Esso può essere utilizzato per alimentare eventuali dispositivi esterni. Infatti, nell'Adapter Board, questa uscita è stata utilizzata per alimentare l'accelerometro. Sia sulla linea di alimentazione del modulo (VLED) sia sull'uscita VLDO sono stati inseriti due condensatori rispettivamente di tipo 0603 e 0402 con valori di \SI{4.7}{\micro\farad} e di \SI{1}{\micro\farad} per filtrare eventuali disturbi sulle alimentazioni. La comunicazione con il microcontrollore avviene tramite i pin SDA e SCL, tipici della comunicazione I\ap{2}C. Il modulo presenta un package di tipo OLGA a 14 pin, dalle dimensioni ridotte, 2.9 mm x 4.3 mm x 1.4 mm, con un consumo tipicamente inferiore a \SI{30}{\micro\watt}.

\paragraph{Accelerometro} Il LIS2DW12 è un accelerometro triassiale lineare ad alte performance e basso consumo, prodotto da STMicroelectronics\cite{STElectronicsLIS2DW12} e realizzato con tecnologia MEMS . 
\begin{figure}[b]
	\centering
	\includegraphics[width=0.3\linewidth]{ImageFiles/Hardware/ImmagineLIS2DW12}
	\caption{Accelerometro LIS2DW12.}
	\label{fig:ImmagineLIS2DW12}
\end{figure}
\`E pensato per rilevazioni di movimento e riconoscimento dei gesti su dispositivi indossabili. Il sensore permette di selezionare un fondo scala di $\pm$\SI{2}{\gram}, $\pm$\SI{4}{\gram}, $\pm$\SI{8}{\gram} oppure $\pm$\SI{16}{\gram}. Richiede un'alimentazione compresa tra i 1.62V e i 3.6V e la corrente assorbita è tipicamente di \SI{50}{\nano\ampere} nella modalità \textit{power-down} e di \SI{1}{\micro\ampere} in quella attiva \textit{low-power}. Inoltre, l'accelerometro include un sensore di temperatura che permette di compensare gli effetti termici sulle misure. Per la comunicazione presenta un'interfaccia I\ap{2}C e un interfaccia SPI. Questo sensore è stato inserito nell'Adapter Board per ottenere informazioni sui movimenti a cui è sottoposto il sensore. Infatti, le acquisizioni fotopletismografiche sono sensibili agli artefatti del movimento, che possono disturbare le misurazioni. Attraverso degli algoritmi particolari è però possibile compensare eventuali errori di misura, permettendo di ottenere dati di qualità superiore. Il sensore è realizzato in un package di tipo LGA-12 con dimensioni 2.0 mm x 2.0 mm x 0.7 mm.

\paragraph{Progetto PCB} Il progetto della board è stato realizzato con il software Autodesk Eagle. Come prima fase del progetto è stato realizzato lo schematico, dove sono state definite le interconnessioni dei componenti a livello logico. In seguito, è stato definito il layout della PCB, definendo la posizione sulla scheda dei componenti e delle piste che li connettono.
In figura \ref{fig:schematic_maxm} viene riportato lo schematico dove si possono vedere il sensore PPG, l'accelerometro e i due connettori necessari per l'alimentazione della scheda (VCC e GND), la comunicazione I\ap{2}C (SDA e SCL) e l'interrupt del modulo PPG (INT\_PPG).
\begin{figure}[b]
	\centering
	\includegraphics[width=0.8\linewidth]{ImageFiles/Hardware/schematic_maxm}
	\caption{Schematico Adapater Board con il sensore PPG MAXM86161}
	\label{fig:schematic_maxm}
\end{figure}
Il layout è stato progettato partendo dallo schematico, avendo cura di realizzare una board dalle dimensioni più ridotte possibili. Come si può vedere in figura \ref{fig:Layout_maxm} la board si compone su due layer, Top (in rosso) e Bottom (in blu).
\begin{figure}[b]
	\centering
	a$)$
	\includegraphics[width=0.1567\linewidth]{ImageFiles/Hardware/layout_top_maxm}
	b$)$
	\includegraphics[width=0.1567\linewidth]{ImageFiles/Hardware/layout_bottom_maxm}
	c$)$
	\includegraphics[width=0.15\linewidth]{ImageFiles/Hardware/manifacturing_top_maxm}
	d$)$
	\includegraphics[width=0.15\linewidth]{ImageFiles/Hardware/manifacturing_bottom_maxm}
	\caption{Layout Adapter Board con modulo MAXM86161: a$)$ Livello Top, b$)$ Livello Bottom, c$)$ Rendering 2D Top, d$)$ Rendering 2D Bottom.}
	\label{fig:Layout_maxm}
\end{figure}
Le piste disegnate hanno una larghezza di \SI{0.2}{\milli\meter} e dei vias con diametro di \SI{0.2}{\milli\meter}, con una corona di \SI{254}{\micro\meter}.

\pagebreak
Nella tabella \ref{tab:ComponentiAdapterMaxm} vengono riassunti i componenti presenti nell'\textit{Adapter Board} descritta.
\begin{table}[h]
	\renewcommand{\arraystretch}{1.5}
	\centering
	\footnotesize
	\begin{tabular}{ccccc}
		\hline Nome    & Tipologia   & Principali caratteristiche   \\ 
		\hline MAXM86161 & Sensore PPG  & \begin{tabular}{@{}c@{}}								
			Dimensioni: 2.9mm x 4.3mm x 1.4mm \\ 
			\hline Package: 14 pin - OLGA \\
			\hline Corrente di shutdown: \SI{1.6}{\micro\ampere} \\
			\hline Tensione di alimentazione: 3.0 - 5.5 V \\
			\hline Tensione di uscita LDO: 1.68 - 1.92 V \\
			\hline Protocollo di comunicazione: I\ap{2}C \\
			\hline LED: Rosso, Infrarosso, Verde
		\end{tabular} \\
		\hline LIS2DW12  & Accelerometro triassiale & 
		\begin{tabular}{@{}c@{}}								
			Dimensioni: 2mm x 2mm x 0.7mm \\ 
			\hline Package: LGA-12 \\
			\hline Consumo di shutdown: \SI{50}{\nano\ampere} \\
			\hline Consumo di corrente: \SI{90}{\micro\ampere} \\
			\hline Tensione di alimentazione: 1.62 - 3.6 V \\
			\hline Fondo scala selezionabile: ±2g | ±4g | ±8g | ±16g \\
			\hline Protocollo di comunicazione: I\ap{2}C \\
			\hline Interrupt programmabili: 2 \\
			\hline Orientamento 6D/4D. 
		\end{tabular} \\ 
		\hline
	\end{tabular}
	\caption{Riepilogo dei componenti dell'Adapter Board con sensore MAXM86161.}
	\label{tab:ComponentiAdapterMaxm}
\end{table}

\pagebreak
\subsection{Adapter Board: MAX86916}
La seconda Adapter Board progettata è costituita da un modulo PPG MAX86916, da un accelerometro LIS2DW12 e da un LDO (Low-dropout regulator) prodotto da Analog Devices. Come nella board descritta precedentemente, il sensore MAX86916 e l'accelerometro comunicano con il controllore grazie a un'interfaccia standard I\ap{2}C (\Fig~\ref{fig:DiagrammaBlocchiMAX86916}). I dati vengono poi acquisiti da un computer collegato tramite USB alla board STM32F4DISCOVERY, che ospita il microcontrollore. L'aggiunta del regolatore LDO esterno e le dimensioni maggiori del modulo PPG ha portato ad ottenere una PCB con dimensioni 4.5 mm x 17.5 mm. Sebbene siano leggermente maggiori rispetto all'Adapter Board sopra descritta, essa risulta essere molto compatta e adatta ad dispositivi indossabili, con una superficie di solo \SI{78.75}{\square\milli\meter}.
\begin{figure}[h]
	\centering
	\includegraphics[width=0.6\linewidth]{ImageFiles/Hardware/DiagrammaBlocchiMAX86916}
	\caption{MAX86916 magari mettere in evidenza led e foto diodo}
	\label{fig:DiagrammaBlocchiMAX86916}
\end{figure}

\paragraph{Sensore PPG} Il sensore PPG utilizzato è il \textbf{MAX86916}, prodotto da Maxim Integrated e rappresentato nella figura\ref{fig:ImmagineMAX86916}.
\begin{figure}[b]
	\centering
	\includegraphics[width=0.6\linewidth]{ImageFiles/Hardware/ImmagineMAX86916}
	\caption{MAX86916 magari mettere in evidenza led e foto diodo}
	\label{fig:ImmagineMAX86916}
\end{figure}
Esso permette l'acquisizione di segnali fotopletismografici grazie a quattro LED, rispettivamente di colore rosso, infrarosso, verde e blu e di un fotodiodo. La peculiarità di questo modulo consiste nella presenza del LED blu, che tipicamente viene omesso. Nel corso di questa tesi, cercheremo di analizzare le caratteristiche delle acquisizioni effettuate con questa lunghezza d'onda (circa \SI{460}{\nano\meter}). Infatti, solitamente vengono utilizzate sorgenti luminose nelle regioni del verde, rosso e infrarosso considerate più efficienti per misure di SpO\ped{2} e HR. I LED vengono alimentati con una tensione di 5.0V \todo{da verificare} fornita dalla scheda STM32F4DISCOVERY, mentre i circuiti interni vengono alimentati con una tensione di 1.8V, grazie al regolatore lineare di tensione. Il modulo si presenta con un package di tipo OLGA a 14 pin con dimensioni 3.5 mm x 7.0 mm x 1.5 mm. Presenta consumi molto ridotti, con una corrente di \SI{0.7}{\micro\ampere} in modalità \textit{shutdown}.

\paragraph{LDO} Per fornire l'alimentazione di al modulo PPG è stato inserito un regolatore lineare di tensione. Il regolatore \textit{low-dropout} è un dispositivo elettronico integrato in grado di fornire una tensione in uscita costante, quando al suo ingresso è applicata un'opportuna tensione\cite{Horowitz2015}.\todo{cit da controllare} In questa Adapter Board, il regolatore viene alimentato con una tensione di 5.0V\todo{da verificare} e fornisce in uscita una tensione di 1.8V che alimenta sia il sensore PPG sia l'accelerometro. Inoltre, sull'ingresso e sull'uscita del LDO sono presenti due condensatori 0402 da \SI{4.7}{\micro\farad} per filtrare eventuali disturbi sulle alimentazioni.

Per la scelta del regolatore da utilizzare in questa Adapter Board, sono stati confrontati diversi LDO disponibili sul mercato e prodotti da Analog Devices: \textbf{ADP166}\cite{AnalogDevicesADP166}, \textbf{ADP122}\cite{AnalogDevicesADP122} e \textbf{ADP151}\cite{AnalogDevicesADP151}. Le loro caratteristiche sono riportarte nella tabella \ref{tab:ConfrontoLDO}
\begin{table}[h]
	\renewcommand{\arraystretch}{1.5}
	\begin{tabular}{cccc}
		\hline
		& \textbf{ADP166} & \textbf{ADP122} & \textbf{ADP151} \\ \hline
		\textbf{Dimensioni {[}mm\ap{2}{]}}                   & 2 x 2           & 2 x 2           & 2 x 2           \\ \hline
		\textbf{Intervallo tensione in ingresso {[}V{]}}  & 2.2 - 5.5       & 2.3 - 5.5       & 2.2 - 5.5       \\ \hline
		\textbf{Intervallo tensione in uscita {[}V{]}}    & 1.0 - 4.2       & 0.8 - 5         & 1.1 - 3.3       \\ \hline
		\textbf{Tensione di dropout {[}mV{]}}            & 120             & 85              & 140             \\ \hline
		\textbf{Rumore in uscita {[}$\mu$Vrms{]}}         & 80              & 65              & 9               \\ \hline
		\textbf{I\ped{q} Corrente di quiesceza {[}$\mu$A{]}}    & 0.89            & 45              & 10              \\ \hline
		\textbf{I\ped{s} Corrente di shutdown {[}nA{]}}      & 50              & 100             & 200             \\ \hline
		\textbf{I\ped{MAX} Massima corrente erogabile {[}mA{]}} & 150             & 300             & 200             \\ \hline
	\end{tabular}
	\caption{Confronto delle caratteristiche di diversi LDO.}
	\label{tab:ConfrontoLDO}
\end{table}

\noindent Il modello scelto è l'ADP166, nella versione con tensione fissa di uscita pari a 1.8V e con un package LFCSP di dimesioni 2 mm x 2 mm. La peculiarità di questo LDO risiede nella bassa corrente di quiescenza, pari a circa \SI{890}{\nano\ampere} con un carico da \SI{1}{\micro\ampere}\cite{AnalogDevicesADP166}. Infatti, la \textit{corrente di quiescenza} è definita come la differenza tra la corrente di ingresso e la corrente di uscita\cite{Lee1999}. In altre parole, essa rappresenta la corrente assorbita dal regolatore di tensione e non trasferita al carico. Per questo motivo, dal momento che si vuole progettare una piattaforma indossabile a bassi consumi, avere una bassa corrente di quiescenza significa diminuire i consumi della Adapter Board. Inoltre, l'integrato presenta una bassa corrente di shutdown, che permette di ridurre i consumi quando esso viene disabilitato attraverso il pin EN. Tuttavia, nella nostra piattaforma abbiamo scelto di mantenere il regolatore sempre attivo.

\paragraph{Accelerometro} L'accelerometro utilizzato è il \textbf{LIS2DH12} prodotto da STMicroelectronics, come nell'Adapter Board precedente.

\paragraph{Progetto PCB} Anche il progetto di questa board è stato realizzato con il software Autodesk Eagle. In figura \ref{fig:schematic_max} viene riportato lo schematico dove si possono vedere il sensore PPG, il regolatore di tensione lineare (LDO), l'accelerometro e i connettori per l'alimentazione della scheda e la comunicazione I\ap{2}C.
\begin{figure}[b]
	\centering
	\includegraphics[width=0.8\linewidth]{ImageFiles/Hardware/schematic_max}
	\caption{Schematico Adapater Board con il sensore PPG MAXM86161.}
	\label{fig:schematic_max}
\end{figure}
La realizzazione del layout è stata fatta sempre avendo cura di avere una scheda dalle dimensioni ridotte. In figura \ref{fig:Layout_maxm} è riportato il layout della board, che si compone su due layer, Top (in rosso) e Bottom (in blu).
\begin{figure}[tb]
	\centering
	a$)$
	\includegraphics[width=0.1459\linewidth]{ImageFiles/Hardware/layout_top_max}
	b$)$
	\includegraphics[width=0.1459\linewidth]{ImageFiles/Hardware/layout_bottom_max}
	c$)$
	\includegraphics[width=0.16\linewidth]{ImageFiles/Hardware/manifacturing_top_max}
	d$)$
	\includegraphics[width=0.16\linewidth]{ImageFiles/Hardware/manifacturing_bottom_max}
	\caption{Layout Adapter Board con modulo MAX86916: a$)$ Livello Top, b$)$ Livello Bottom, c$)$ Rendering 2D Top, d$)$ Rendering 2D Bottom.}
	\label{fig:Layout_max}
\end{figure}

\noindent Le piste disegnate hanno una larghezza di 0.2 mm e dei vias con diametro di 0.2, con una corona di \SI{254}{\micro\meter}.
\pagebreak

\noindent Nella tabella \ref{tab:ComponentiAdapterMax} vengono riassunti i componenti presenti nell'\textit{Adapter Board} descritta.
\begin{table}[h]
	\renewcommand{\arraystretch}{1.5}
	\centering
	\footnotesize
	\begin{tabular}{ccccc}
		\hline Nome    & Tipologia   & Principali caratteristiche   \\ 
		\hline MAX86916 & Sensore PPG  & \begin{tabular}{@{}c@{}}								
			Dimensioni: 3.5mm x 7.0mm x 1.5mm, \\ 
			\hline Package: 14 pin - OLGA \\
			\hline Corrente di shutdown: \SI{0.7}{\micro\ampere} \\
			\hline Tensione di alimentazione PPG: 1.7 - 2.0 V \\
			\hline Tensione di alimentazione LED: 3.5 - 5.5 V \\
			\hline Protocollo di comunicazione: I\ap{2}C \\
			\hline LED: Rosso, Infrarosso, Verde, Blu
		\end{tabular} \\
		\hline ADP166  & Regolatore di tensione lineare (LDO) & \begin{tabular}{@{}c@{}}								
			Dimensioni: 2.0mm x 2.0mm x 1.5mm, \\ 
			\hline Package: 6-lead LFCSP \\
			\hline Corrente di quiescenza: \SI{590}{\nano\ampere} - \SI{890}{\nano\ampere} \\
			\hline Corrente di shutdown: \SI{50}{\nano\ampere} \\
			\hline Massima corrente operativa: \SI{150}{\milli\ampere} \\
			\hline Tensione di ingresso: 2.2 - 5.5 V \\
			\hline Tensione di dropout: \SI{120}{\milli\volt} \\
			\hline Rumore in uscita: \textcolor{red}{\SI{100}{\micro\volt}rms} \\ 
		\end{tabular} \\
		\hline LIS2DW12  & Accelerometro triassiale & vedi sopra \\
	\end{tabular}
	\caption{Riepilogo dei componenti dell'Adapter Board con sensore MAX86916.}
	\label{tab:ComponentiAdapterMax}
\end{table}
\todo{da verificare il rumore e aggiungere l'accelerometro e non vedi sopra}

\pagebreak
\subsection{Microcontrollore: STM32F4DISCOVERY}
La comunicazione con i moduli PPG e l'accelerometro avviene tramite il protocollo I\ap{2}C. Per permettere a un computer di leggere, memorizzare e analizzare i dati raccolti dei sensori è necessario utilizzare un microcontrollore che supporti la comunicazione I\ap{2}C e permetta di inviare dati, tramite USB, a un PC. Il \textit{discovery kit} STM32F4DISCOVERY\cite{STMicroelectronics2020}, rappresentata in figura \ref{fig:ImmagineSTM32F4DISCOVERY}, è una board prodotta da STMicroelectronics che permette di semplificare il collegamento e l'utilizzo di un microcontrollore. In particolare, essa integra un microntrollore STM32F407 ad alte prestazioni prodotto dalla STMicroelectronics. In aggiunta, sulla scheda sono presenti un accelerometro digitale MEMS, un microfono omnidirezionale MEMS, pulsanti e LED utili che possono essere programmati e utilizzati dall'utente. Grazie a pin specifici\todo{quali pin usereremo?} è possibile collegare le Adapter Board progettate e utilizzando la USB comunicare i dati ad un PC collegato,\todo{cosa useremo per comunicare con il mirco?}. Inoltre, la board permette l'alimentazioni delle board tramite i pin GND e 5V.\todo{alimenteremo entrambe le board a 5V?}
\begin{figure}[h]
	\centering
	\includegraphics[width=0.5\linewidth]{ImageFiles/Hardware/ImmagineSTM32F4DISCOVERY}
	\caption{Board STM32F4DISCOVERY.}
	\label{fig:ImmagineSTM32F4DISCOVERY}
\end{figure}

\clearpage