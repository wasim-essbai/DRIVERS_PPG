\section{Siti di misura e lunghezze d'onda impiegate}
\subsection{Assorbimento dell'energia luminosa e lunghezze d'onda}
\textcolor{red}{La pelle è la superficie di interazione tra il corpo umano e l'ambiente esterno. Uno dei fattori ambientali con cui interagisce è la luce. L'analisi dell'interazione luce-pelle presenta alcune criticità, a causa dei complessi fenomeni di diffusione, assorbimento e riflessione. L'assorbimento e la diffusione delle onde luminose contribuiscono in modo significativo all'aspetto della pelle. Si stima che circa una percentuale dal 4\% al 7\% della luce visibile sia riflessa dalla superficie epidermica, indipendentemente dalla lunghezza d'onda e del colore della pelle. La restante parte viene rifratta e assorbita dalla pelle (dal articolo citato adesso)}
Il fenomeno dell'assorbimento rappresenta una riduzione dell'energia luminosa. Nei tessuti umani questo è dovuto principalmente a due sostanze \cite{Lister2012}: l'emoglobina e la melanina. 
L'emoglobina, spesso indicata con la sigla Hb, è una proteina contenente ferro in grado di combinarsi in modo reversibile con l’ossigeno molecolare \cite{SilverthornDeeUnglaub2020Fu:u}. Essa presenta tre picchi di assorbimento nella regione della luce visibile. Il primo (noto anche come Picco di Soret) si trova all'interno della regione blu dello spettro ed è dominante. Gli altri due invece si possono distinguere nella regione giallo-verde, con una lunghezza d'onda compresa tra i 500 e 600 nm. Questi tre picchi combinati danno all'emoglobina un colore rosso. \textcolor{red}{qui farei riferimento all'immagine condivisa sul team (presa dall'articolo allegato) che fa vedere i tre picchi. La inserirei dopo visto che presenta anche gli spettri delle altre sostanze)}
La melanina è contenuta sotto forma di granuli nelle cellule dello strato basale dell’epidermide \cite{SilverthornDeeUnglaub2020Fu:u}. Essa assorbe principalmente le lunghezze d'onda più corte, quindi presenta uno spettro di assorbimento che decresce gradualmente dall'ultravioletto fino all'infrarosso. In realtà, la melanina è una molecola molto complessa e la sua struttura non è ancora ben nota. \textcolor{red}{solito riferimento alla figura}
Un altro elemento di cui tener conto è l'acqua che presenta un basso assorbimento nella regione visibile, mentre assorbe la luce nel regime ultravioletto e del lontano infrarosso. Le regioni di luce che riescono ad attraversare maggiormente i tessuti umani sono quindi quella rossa e del vicino infrarosso (\Fig~\ref{fig:FlussoParabolico}). Per questo motivo sono spesso utilizzate per applicazioni fotopletismografiche.

\begin{figure}[h]
	\centering
	\includegraphics[width=0.7\linewidth]{ImageFiles/Fotopletismografia/PenetrazioneLuce}
	\caption{Profondità nel tessuto raggiunta dalla luce in funzione della lunghezza d'onda.}
	\label{fig:PenetrazioneLuce}
\end{figure}

La profondità nel tessuto umano che si può raggiungere dipende dalla lunghezza d'onda ma anche dalla distanza tra la sorgente luminosa e il foto-rilevatore.\todo{ok che dipende dalla distanza... ma o spieghi qualcosa o mi sembra un po' buttata li questa cosa} \textcolor{red}{Come evidenziato prima}, le luce rossa e infrarossa permettono acquisizioni a maggiore profondità nel tessuto. Al contrario la luce verde viene assorbita in quantità superiore\cite{Lee2021} a causa dell'emoglobina e della deossiemoglobina \todo{ancora riferimento all'immagine che lo dice}. Infatti, nelle zone in cui il sangue pulsa attraverso la pelle si rileva una variazione tra luce emessa e riflessa maggiore per la regione verde rispetto a quella infrarossa. Questa caratteristica permette di avere un rapporto segnale-rumore (Signal to Noise Ratio), superiore a quello ottenuto con luce infrarossa. Pertanto, la luce verde risulta più adatta per misure superficiali come la variazione del flusso sanguigno superficiale \cite{Youssef2020}.
\subsection{Siti di misura}
La scelta del sito di misura risulta ancora oggi oggetto di discussione dal momento che è difficile definire una zona del corpo "migliore" per acquisizioni fotopletismografiche. In realtà, la scelta va fatta in base alla tipologia di applicazione e alla qualità dell'acquisizione che si reputa accettabile.
\begin{figure}[h]
	\centering
	\includegraphics[width=0.7\linewidth]{ImageFiles/Fotopletismografia/ZoneAcquisizione}
	\caption{I principali siti di misura impiegati per segnali PPG.}
	\label{fig:ZoneAcquisizione}
\end{figure}
Tuttavia ci sono alcune zone che vengono privilegiati per il posizionamento dei sensori PPG.
	\paragraph{Polpastrello}
	 Rappresentano la zona più comune per acquisizioni e costituiscono un'area ricca di capillari e con una bassa concentrazione di melanina. Per questo motivo è particolarmente adatta a misure in profondità con luce rossa e infrarossa, che permettono misure della frequenza cardiaca e di saturazione di emoglobina
	 \paragraph{Polso}\todo{se non sbaglio vedo forte contraddizioni con la tesi di facagni}
	 Un primo tipo di acquisizioni sul polso utilizza la parte superiore (postero-esterno). In questo punto si ha maggiore presenza di capillari sanguigni. Per questo tipo di misura si impiega principalmente la luce verde.\todo{non trovo un riferimento... perchè usiamo la luce verde?????}
	 Il secondo tipo di acquisizione invece è quello che coinvolge la parte inferiore(antero-interna), dove si ha il passaggio delle arterie ulnare e radiale. Risultano quindi adatte le luci rossa e infrarossa \todo{vedi tesi facagni... diversa}. Questa tipologia di acquisizione permette l'ottenimento di risultati migliori e più accurati.
	 Tuttavia, nei dispositivi commerciali, prevale l'acquisizione sulla faccia postero-esterna poiché sono più agevoli da indossare.
	 Un limite di queste misure è la presenza di disturbi legati al movimento del braccio che però possono essere ridotti tramite l'integrazione di una piattaforma inerziale nel sensore PPG \cite{Ghamari2018}.
	 \paragraph{Fronte}
	 Si tratta di una regione ricca di arterie che si diramano dalla carotide interna \cite{Abay2019}. Per cui le misure in questa zona offrono quindi una buona qualità e affidabilità del segnale. Le acquisizioni sulla fronte sono anche meno sensibili ai disturbi legati al movimento ed è una posizione più accessibile rispetto ai polpastrelli. Questo sito viene utilizzato tipicamente per la misura della saturazione arteriosa di ossigeno. \todo{vedo una non corrispondenza con facagni}
 	 \paragraph{Orecchio}
	 I lobi delle orecchie rappresentano una zona con basso contenuto di cartilagine e elevata affluenza di sangue. Queste caratteristiche permettono di ottenere dei segnali di buona qualità. Anche per il lobi delle orecchie si ha meno vulnerabilità agli artefatti dovuti al movimento e offrono una posizione comoda. Infatti, i dispositivi di questo tipo sono altamente indossabili \cite{Ghamari2018}\cite{Vescio2018}.\todo{domanda.. necessarie due citazioni}

 